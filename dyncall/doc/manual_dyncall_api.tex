\newpage
\section{\emph{Dyncall} C library API}

The library provides low-level functionality to make foreign function calls
from different run-time environments. The flexibility is constrained by the
set of supported types.

\paragraph{C interface style conventions}

This manual and the \product{dyncall} library's C interface {\tt "dyncall.h"}
uses the following C source code style.


\begin{table}[h]
\begin{center}
\begin{tabular*}{0.8\textwidth}{llll}
\hline
Subject    & C symbol & Details & Example \\
\hline  
Types      
  & {\tt DC\group{type name}}      
  & lower-case & \capi{DCint}, \capi{DCfloat}, \capi{DClong}, \ldots\\
Structures 
  & {\tt DC\group{structure name}} 
  & camel-case 
  & \capi{DCCallVM}\\
Functions  & {\tt dc\group{function name}}  & camel-case & \capi{dcNewCallVM}, \capi{dcArgInt}, \ldots\\
\hline
\end{tabular*}
\caption{C interface conventions}
\label{sourecode}
\end{center}
\end{table}

\subsection{Supported C/C++ argument and return types}

\begin{table}[h]
\begin{center}
\begin{tabular*}{0.75\textwidth}{ll}
\hline
Type alias & C/C++ data type\\
\hline
DCbool  & \_Bool, bool\\
DCchar  & char\\
DCshort & short\\
DCint   & int\\
DClong  & long\\
DClonglong & long long\\
DCfloat & float\\
DCdouble & double\\
DCpointer & void*\\
DCvoid    & void\\
\hline
\end{tabular*}
\caption{Supported C/C++ argument and return types}
\label{types}
\end{center}
\end{table}

\pagebreak

\subsection{Call Virtual Machine - CallVM}

This \emph{CallVM} is the main entry to the functionality of the library.

\paragraph{Types}

\begin{lstlisting}[language=c]
typedef void DCCallVM; /* abstract handle */
\end{lstlisting}

\paragraph{Details}
The \emph{CallVM} is a state machine that manages all aspects of a function 
call from configuration, argument passing up the actual function call on
the processor.

\subsection{Allocation}

\paragraph{Functions}

\begin{lstlisting}[language=c]
DCCallVM* dcNewCallVM (DCsize size);
void      dcFreeCallVM(DCCallVM* vm);
\end{lstlisting}

\lstinline{size} specifies the amount of memory that will be allocated for
the preparation of the function call in the binding phase (\ref{phases}).

% \paragraph{Explicit allocation}
% \capi{dcGetSizeOfCallVM()}
% \capi{dcInitCallVM()}
% \capi{dcShutdownCallVM()}

\subsection{Configuration}

\paragraph{Function}

\begin{lstlisting}[language=c]
void dcMode (DCCallVM* vm, DCint mode);
\end{lstlisting}

\paragraph{Modes}

\begin{table}[h]
\begin{center}
\begin{tabular*}{0.75\textwidth}{ll}
\hline
Constant & Description\\
\hline
\lstinline@DC_CALL_C_DEFAULT@            & Default C function call\\
\lstinline@DC_CALL_C_X86_WIN32_STD@      & X86 Windows standard call\\
\lstinline@DC_CALL_C_X86_WIN32_FAST_MS@  & X86 Windows Microsoft fast call\\
\lstinline@DC_CALL_C_X86_WIN32_THIS_MS@  & X86 Windows Microsoft this call\\
\lstinline@DC_CALL_C_X86_WIN32_THIS_GNU@ & X86 Windows MinGW this call\\
\lstinline@DC_CALL_C_X86_WIN32_FAST_GNU@ & X86 Windows MinGW fast call\\
\lstinline@DC_CALL_C_X64_WIN64@          & X64 Windows standard C call\\
\lstinline@DC_CALL_C_PPC32_DARWIN@       & PPC32 Mac OS X standard C call\\
\lstinline@DC_CALL_ARM@                  & ARM standard call\\
\hline
\end{tabular*}
\caption{CallVM calling convention modes}
\label{functioncalls}
\end{center}
\end{table}

\paragraph{Details}

\lstinline@DC_CALL_C_DEFAULT@ is the default standard C call on the target platform.
On most platforms, there is only one C calling convention. Only the X86 platform
provides a rich family of different calling conventions.


\subsection{Machine state reset}

\begin{lstlisting}[language=c]
void dcReset(DCCallVM* vm);
\end{lstlisting}

Resets the internal stack of arguments. This function should be called prior 
to binding new arguments to the CallVM, because arguments don't get flushed
automatically after a call invocation.


\subsection{Argument forwarding}

\paragraph{Functions}

\begin{lstlisting}[language=c]
void dcArgBool    (DCCallVM* vm, DCbool     arg);
void dcArgChar    (DCCallVM* vm, DCchar     arg);
void dcArgShort   (DCCallVM* vm, DCshort    arg);
void dcArgInt     (DCCallVM* vm, DCint      arg);
void dcArgLong    (DCCallVM* vm, DClong     arg);
void dcArgLongLong(DCCallVM* vm, DClonglong arg);
void dcArgFloat   (DCCallVM* vm, DCfloat    arg);
void dcArgDouble  (DCCallVM* vm, DCdouble   arg);
void dcArgPointer (DCCallVM* vm, DCpointer  arg);
\end{lstlisting}

\paragraph{Details}

Arguments should be bound in \emph{left-to-right} order.

\subsection{Call invocation}

\paragraph{Functions}

\begin{lstlisting}[language=c]
DCvoid     dcCallVoid    (DCCallVM* vm, DCpointer funcptr);
DCbool     dcCallBool    (DCCallVM* vm, DCpointer funcptr);
DCchar     dcCallChar    (DCCallVM* vm, DCpointer funcptr);
DCshort    dcCallShort   (DCCallVM* vm, DCpointer funcptr);
DCint      dcCallInt     (DCCallVM* vm, DCpointer funcptr);
DClong     dcCallLong    (DCCallVM* vm, DCpointer funcptr);
DClonglong dcCallLongLong(DCCallVM* vm, DCpointer funcptr);
DCfloat    dcCallFloat   (DCCallVM* vm, DCpointer funcptr);
DCdouble   dcCallDouble  (DCCallVM* vm, DCpointer funcptr);
DCpointer  dcCallPointer (DCCallVM* vm, DCpointer funcptr);
\end{lstlisting}

\paragraph{Details}

After the invocation of the call, the arguments are still bounded and a second call using the
same arguments can be issued. If you need to clear the argument bindings, 
you have to reset the \emph{CallVM}.

\pagebreak

\subsection{Formatted calls (ANSI C ellipsis interface)}

\paragraph{Functions}

\begin{lstlisting}[language=c]
void dcCallF (DCCallVM* vm, DCValue* result, DCpointer funcptr, 
              const DCsigchar* signature, ...);
void dcVCallF(DCCallVM* vm, DCValue* result, DCpointer funcptr, 
              const DCsigchar* signature, va_list args);
\end{lstlisting}



