\newpage
\section{Building the library}

The library has been built and used successfully on several 
platform/architecture configurations and build systems.
Please see notes on specfic platforms to check if the target
architecture is currently supported.


\subsection{Requirements}

The following tools are supported directly to build the \product{dyncall} library.
However, as the number of source files to be compiled for a given
platform is small, it shouldn't be difficult to build it manually with
another toolchain.
\begin{itemize}
\item C compiler to build the \product{dyncall} library (GCC or Microsoft C/C++ compiler)
\item C++ compiler to build the optional test cases (GCC or Microsoft C/C++ compiler)
\item Python (optional - for generation of some test cases)
\item BSD make, GNU make, or Microsoft nmake as automated build tools
\end{itemize}


\subsection{Supported/tested platforms and build systems}

Although it is possible to build the \product{dyncall} library on more platforms
than the ones outlined here, this section doesn't list operating systems or
architectures the authors didn't test. However, untested platforms using
the same build tools (e.g. the BSD family of operating systems using similar
flavors of the BSD make utility along with GCC, etc.) should work without
modification. If you have problems building the \product{dyncall} library on one of the
platforms mentioned below, or if you successfully built it on a yet unlisted
one, please let us know.\\
\\

\begin{tabular}{l l}
{\bf{\large x86}} &                                           \\
\hline\hline
{\bf Windows}     & nmake, GNU make (via MinGW)               \\
{\bf Darwin}      & GNU make, BSD make                        \\
{\bf Linux}       & GNU make                                  \\
{\bf SunOS}       & GNU make (tested on GNU-centric NexentaOS)\\
{\bf FreeBSD}     & BSD make                                  \\
{\bf OpenBSD}     & BSD make                                  \\
\hline
                  &                                           \\
                  &                                           \\


{\bf{\large x64}} &                                           \\
\hline\hline
{\bf Windows}     & nmake                                     \\
{\bf OpenBSD}     & BSD make                                  \\
\hline
                  &                                           \\
                  &                                           \\


{\bf{\large PowerPC (32bit)}} &                               \\
\hline\hline
{\bf Darwin}                  & GNU make, BSD make            \\
\hline
                  &                                           \\
                  &                                           \\


{\bf{\large ARM9E (ARM mode)}} &                             \\
\hline\hline
{\bf Nintendo DS} & ARMv5TE (of ARM9E family) binary manually built with \cite{devkitPro}\\
\hline
                  &                                           \\
                  &                                           \\
		  
{\bf{\large MIPS32}} &              \\
\hline\hline
{\bf Playstation Portable} & GNU make                        \\
\hline

\end{tabular}\\

\pagebreak

\subsection{Build instructions}


\begin{enumerate}
\item Configure the source

\paragraph{*nix flavour}
\begin{lstlisting}
./configure [options]
\end{lstlisting}

Available options:

\begin{tabular}{ll}	
{\tt --prefix path } & specify installation prefix \\
{\tt --target-x86 } & build for x86 architecture \\
{\tt --target-x64 } & build for x64 architecture \\
{\tt --target-psp } & cross-compile build for Playstation Portable \\
{\tt --target-nds } & cross-compile build for Nintendo DS \\
{\tt --tool-gcc } & use GNU Compiler Collection tool-chain \\
{\tt --tool-msvc } & use Microsoft Visual C++ \\
{\tt --asm-nasm } & use NASM Assembler \\
{\tt --asm-ml } & use Microsoft Macro Assembler \\
\end{tabular}


\paragraph{windows flavour}

\begin{lstlisting}
.\configure [options]
\end{lstlisting}

\begin{tabular}{ll}
{\tt /x86 } & build for x86 architecture \\
{\tt /x64 } & build for x64 architecture \\
\end{tabular}

\item Build the static libraries \product{dyncall} and \product{dynload}
\begin{lstlisting}
make                              # using {GNU,BSD} Make
nmake /f Nmakefile                # using NMake on Windows
\end{lstlisting}
\item Install libraries and includes
\begin{lstlisting}
make install 
\end{lstlisting}
\item Optionally, build the test suites (Python required)
\begin{lstlisting}
make test
name /f Namekfile test
\end{lstlisting}
\end{enumerate}
