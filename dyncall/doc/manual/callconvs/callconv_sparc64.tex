%//////////////////////////////////////////////////////////////////////////////
%
% Copyright (c) 2012 Daniel Adler <dadler@uni-goettingen.de>, 
%                    Tassilo Philipp <tphilipp@potion-studios.com>
%
% Permission to use, copy, modify, and distribute this software for any
% purpose with or without fee is hereby granted, provided that the above
% copyright notice and this permission notice appear in all copies.
%
% THE SOFTWARE IS PROVIDED "AS IS" AND THE AUTHOR DISCLAIMS ALL WARRANTIES
% WITH REGARD TO THIS SOFTWARE INCLUDING ALL IMPLIED WARRANTIES OF
% MERCHANTABILITY AND FITNESS. IN NO EVENT SHALL THE AUTHOR BE LIABLE FOR
% ANY SPECIAL, DIRECT, INDIRECT, OR CONSEQUENTIAL DAMAGES OR ANY DAMAGES
% WHATSOEVER RESULTING FROM LOSS OF USE, DATA OR PROFITS, WHETHER IN AN
% ACTION OF CONTRACT, NEGLIGENCE OR OTHER TORTIOUS ACTION, ARISING OUT OF
% OR IN CONNECTION WITH THE USE OR PERFORMANCE OF THIS SOFTWARE.
%
%//////////////////////////////////////////////////////////////////////////////

\subsection{SPARC64 Calling Convention}

\paragraph{Overview}

The SPARC family of processors is based on the SPARC instruction set architecture, which comes in basically tree revisions,
V7, V8 and V9. The former two are 32-bit (see previous chapter) whereas the latter refers to the 64-bit SPARC architecture. SPARC is big endian.

\paragraph{\product{dyncall} support}

\product{dyncall} fully supports the SPARC 64-bit instruction set (V9), \product{dyncallback} support is missing, though.

\subsubsection{SPARC (64-bit) Calling Convention}

@@@ finish
