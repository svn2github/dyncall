%//////////////////////////////////////////////////////////////////////////////
%
% Copyright (c) 2007,2009 Daniel Adler <dadler@uni-goettingen.de>, 
%                         Tassilo Philipp <tphilipp@potion-studios.com>
%
% Permission to use, copy, modify, and distribute this software for any
% purpose with or without fee is hereby granted, provided that the above
% copyright notice and this permission notice appear in all copies.
%
% THE SOFTWARE IS PROVIDED "AS IS" AND THE AUTHOR DISCLAIMS ALL WARRANTIES
% WITH REGARD TO THIS SOFTWARE INCLUDING ALL IMPLIED WARRANTIES OF
% MERCHANTABILITY AND FITNESS. IN NO EVENT SHALL THE AUTHOR BE LIABLE FOR
% ANY SPECIAL, DIRECT, INDIRECT, OR CONSEQUENTIAL DAMAGES OR ANY DAMAGES
% WHATSOEVER RESULTING FROM LOSS OF USE, DATA OR PROFITS, WHETHER IN AN
% ACTION OF CONTRACT, NEGLIGENCE OR OTHER TORTIOUS ACTION, ARISING OUT OF
% OR IN CONNECTION WITH THE USE OR PERFORMANCE OF THIS SOFTWARE.
%
%//////////////////////////////////////////////////////////////////////////////

% ==================================================
% PowerPC 64
% ==================================================
\subsection{PowerPC (64bit) Calling Convention}



\paragraph{Overview}

For in-depth details about the PowerPC (64bit) calling convention, take a look
at Apple's official documentation \cite{ppcMacOSX}.
@@@


\paragraph{\product{dyncall} support}

\product{dyncall} supports PPC64 for calls and callbacks on little and big endian platforms.


\paragraph{Registers and register usage}

64bit PPC: only low 32 bits are saved and preserved


\paragraph{Parameter passing}

@@@


\paragraph{Return values}

@@@


\paragraph{Stack layout}

@@@

