%
% Copyright (c) 2014,2015 Daniel Adler <dadler@uni-goettingen.de>, 
%                         Tassilo Philipp <tphilipp@potion-studios.com>
%
% Permission to use, copy, modify, and distribute this software for any
% purpose with or without fee is hereby granted, provided that the above
% copyright notice and this permission notice appear in all copies.
%
% THE SOFTWARE IS PROVIDED "AS IS" AND THE AUTHOR DISCLAIMS ALL WARRANTIES
% WITH REGARD TO THIS SOFTWARE INCLUDING ALL IMPLIED WARRANTIES OF
% MERCHANTABILITY AND FITNESS. IN NO EVENT SHALL THE AUTHOR BE LIABLE FOR
% ANY SPECIAL, DIRECT, INDIRECT, OR CONSEQUENTIAL DAMAGES OR ANY DAMAGES
% WHATSOEVER RESULTING FROM LOSS OF USE, DATA OR PROFITS, WHETHER IN AN
% ACTION OF CONTRACT, NEGLIGENCE OR OTHER TORTIOUS ACTION, ARISING OUT OF
% OR IN CONNECTION WITH THE USE OR PERFORMANCE OF THIS SOFTWARE.
%

% ==================================================
% ARM64
% ==================================================
\subsection{ARM64 Calling Convention}

\paragraph{Overview}

ARMv8 introduced the AArch64 calling convention.
The word size is 64 bits.\\
For more details, take a look at the Procedure Call Standard for the ARM 64-bit Architecture \cite{AArch64}.


\paragraph{\product{dyncall} support}

The \product{dyncall} library supports AArch64. @@@


\paragraph{Registers and register usage}

\begin{table}[h]
\begin{tabular}{3 B}
\hline
Name         & Brief description\\
\hline
{\bf ...}     & fill in... @@@\\
\hline
\end{tabular}
\caption{Register usage on arm64}
\end{table}

\paragraph{Parameter passing}

\begin{itemize}
\item @@@ fill in
\end{itemize}

\paragraph{Return values}
\begin{itemize}
\item @@@ fill in
\end{itemize}

\paragraph{Stack layout}

Stack directly after function prolog:\\

\begin{figure}[h]
\begin{tabular}{5|3|1 1}
                                         & \vdots &                                      &                              \\
%@@@ fill in
\end{tabular}
\caption{Stack layout on arm64}
\end{figure}

\newpage

