% ==================================================
% x86 
% ==================================================
\subsection{x86 Calling Conventions}


\paragraph{Overview}

There are numerous different calling conventions on the x86 processor
architecture, like cdecl, MS fastcall, GNU fastcall, Borland fastcall, Watcom
fastcall, Win32 stdcall, MS thiscall, GNU thiscall and the pascal calling
convention, etc\ldots


\paragraph{\product{dyncall} support}

Currently, only cdecl, MS fastcall, MS thiscall and GNU thiscall are
supported.\\
\\


\subsubsection{cdecl}

\paragraph{Registers and register usage}

\begin{table}[h]
\begin{tabular}{3 B}
\hline
Name          & Brief description\\
\hline
{\bf eax}     & scratch, return value\\
{\bf ebx}     & permanent\\
{\bf ecx}     & scratch\\
{\bf edx}     & scratch, return value\\
{\bf esi}     & permanent\\
{\bf edi}     & permanent\\
{\bf ebp}     & permanent\\
{\bf esp}     & stack pointer\\
{\bf st0}     & scratch, floating point return value\\
{\bf st1-st7} & scratch\\
\hline
\end{tabular}
\caption{Register usage on x86 cdecl calling convention}
\end{table}

\paragraph{Parameter passing}

\begin{itemize}
\item stack parameter order: right-to-left
\item caller cleans up the stack@@@ doesn't belong to "parameter passing"
\item all parameters are pushed onto the stack
\item stack is usually 4 byte aligned (GCC \textgreater=\ 3.x seems to use a 16byte alignement)
\end{itemize}


\paragraph{Return values}

\begin{itemize}
\item return values of pointer or integral type (\textless=\ 32 bits) are returned via the eax register
\item integers \textgreater\ 32 bits are returned via the eax and edx registers
\item floating point types are returned via the st0 register
\end{itemize}


\paragraph{Stack layout}

\begin{figure}[h]
\begin{tabular}{5|3|1 1}
\hhline{~-~~}
                                  & \vdots                     &                                &                              \\
\hhline{~=~~}
local data                        &                            &                                & \mrrbrace{5}{caller's frame} \\
\hhline{~-~~}
\mrlbrace{3}{parameter area}      & \ldots                     & \mrrbrace{3}{stack parameters} &                              \\
                                  & \ldots                     &                                &                              \\
                                  & \ldots                     &                                &                              \\
\hhline{~-~~}
                                  & return address             &                                &                              \\
\hhline{~=~~}
local data                        &                            &                                & \mrrbrace{3}{current frame}  \\
\hhline{~-~~}
parameter area                    &                            &                                &                              \\
\hhline{~-~~}
                                  & \vdots                     &                                &                              \\
\hhline{~-~~}
\end{tabular}
\\
\\
\\
\caption{Stack layout on x86 cdecl calling convention}
\end{figure}

\subsubsection{MS fastcall}

\paragraph{Registers and register usage}

\begin{table}[h]
\begin{tabular}{3 B}
\hline
Name          & Brief description\\
\hline
{\bf eax}     & scratch, return value\\
{\bf ebx}     & permanent\\
{\bf ecx}     & scratch, parameter 0\\
{\bf edx}     & scratch, parameter 1, return value\\
{\bf esi}     & permanent\\
{\bf edi}     & permanent\\
{\bf ebp}     & permanent\\
{\bf esp}     & stack pointer\\
{\bf st0}     & scratch, floating point return value\\
{\bf st1-st7} & scratch\\
\hline
\end{tabular}
\caption{Register usage on x86 MS fastcall calling convention}
\end{table}

\paragraph{Parameter passing}

\begin{itemize}
\item stack parameter order: right-to-left
\item called function cleans up the stack
\item first two integers/pointers (\textless=\ 32bit) are passed via ecx and edx (even if preceded by other arguments)
\item integer types 64 bits in size @@@ ? first in edx:eax ?
\item all other parameters are pushed onto the stack
\end{itemize}

\paragraph{Return values}

\begin{itemize}
\item return values of pointer or integral type (\textless=\ 32 bits) are returned via the eax register
\item integers \textgreater\ 32 bits are returned via the eax and edx registers@@@verify
\item floating point types are returned via the st0 register@@@ really ?
\end{itemize}


\paragraph{Stack layout}

\begin{figure}[h]
\begin{tabular}{5|3|1 1}
\hhline{~-~~}
                                  & \vdots                     &                                &                              \\
\hhline{~=~~}
local data                        &                            &                                & \mrrbrace{5}{caller's frame} \\
\hhline{~-~~}
\mrlbrace{3}{parameter area}      & \ldots                     & \mrrbrace{3}{stack parameters} &                              \\
                                  & \ldots                     &                                &                              \\
                                  & \ldots                     &                                &                              \\
\hhline{~-~~}
                                  & return address             &                                &                              \\
\hhline{~=~~}
local data                        &                            &                                & \mrrbrace{3}{current frame}  \\
\hhline{~-~~}
parameter area                    &                            &                                &                              \\
\hhline{~-~~}
                                  & \vdots                     &                                &                              \\
\hhline{~-~~}
\end{tabular}
\caption{Stack layout on x86 MS fastcall calling convention}
\end{figure}



\subsubsection{GNU fastcall}

\paragraph{Registers and register usage}

\begin{table}[h]
\begin{tabular}{3 B}
\hline
Name          & Brief description\\
\hline
{\bf eax}     & scratch, return value\\
{\bf ebx}     & permanent\\
{\bf ecx}     & scratch, parameter 0\\
{\bf edx}     & scratch, parameter 1, return value\\
{\bf esi}     & permanent\\
{\bf edi}     & permanent\\
{\bf ebp}     & permanent\\
{\bf esp}     & stack pointer\\
{\bf st0}     & scratch, floating point return value\\
{\bf st1-st7} & scratch\\
\hline
\end{tabular}
\caption{Register usage on x86 GNU fastcall calling convention}
\end{table}

\paragraph{Parameter passing}

\begin{itemize}
\item stack parameter order: right-to-left
\item called function cleans up the stack
\item if the first one or two arguments are integers/pointers (\textless=\ 32bit) , they are passed via ecx and edx
\item integer types 64 bits in size @@@ ? first in edx:eax ?
\item all other parameters are pushed onto the stack
\end{itemize}


\paragraph{Return values}

\begin{itemize}
\item return values of pointer or integral type (\textless=\ 32 bits) are returned via the eax register
\item integers \textgreater\ 32 bits are returned via the eax and edx registers
\item floating point types are returned via the st0 register@@@ really ?
\end{itemize}


\paragraph{Stack layout}

\begin{figure}[h]
\begin{tabular}{5|3|1 1}
\hhline{~-~~}
                                  & \vdots                     &                                &                              \\
\hhline{~=~~}
local data                        &                            &                                & \mrrbrace{5}{caller's frame} \\
\hhline{~-~~}
\mrlbrace{3}{parameter area}      & \ldots                     & \mrrbrace{3}{stack parameters} &                              \\
                                  & \ldots                     &                                &                              \\
                                  & \ldots                     &                                &                              \\
\hhline{~-~~}
                                  & return address             &                                &                              \\
\hhline{~=~~}
local data                        &                            &                                & \mrrbrace{3}{current frame}  \\
\hhline{~-~~}
parameter area                    &                            &                                &                              \\
\hhline{~-~~}
                                  & \vdots                     &                                &                              \\
\hhline{~-~~}
\end{tabular}
\caption{Stack layout on GNU fastcall calling convention}
\end{figure}


\pagebreak

\subsubsection{Borland fastcall}

\paragraph{Registers and register usage}

\begin{table}[h]
\begin{tabular}{3 B}
\hline
Name          & Brief description\\
\hline
{\bf eax}     & scratch, parameter 0, return value\\
{\bf ebx}     & permanent\\
{\bf ecx}     & scratch, parameter 2\\
{\bf edx}     & scratch, parameter 1, return value\\
{\bf esi}     & permanent\\
{\bf edi}     & permanent\\
{\bf ebp}     & permanent\\
{\bf esp}     & stack pointer\\
{\bf st0}     & scratch, floating point return value\\
{\bf st1-st7} & scratch\\
\hline
\end{tabular}
\caption{Register usage on Borland fastcall calling convention}
\end{table}

\paragraph{Parameter passing}

\begin{itemize}
\item stack parameter order: left-to-right
\item called function cleans up the stack
\item first three integers/pointers (\textless=\ 32bit) are passed via eax, ecx and edx (even if preceded by other arguments@@@?)
\item integer types 64 bits in size @@@ ?
\item all other parameters are pushed onto the stack
\end{itemize}


\paragraph{Return values}

\begin{itemize}
\item return values of pointer or integral type (\textless=\ 32 bits) are returned via the eax register
\item integers \textgreater\ 32 bits are returned via the eax and edx registers@@@ verify
\item floating point types are returned via the st0 register@@@ really ?
\end{itemize}


\pagebreak

\paragraph{Stack layout}

\begin{figure}[h]
\begin{tabular}{5|3|1 1}
\hhline{~-~~}
                                  & \vdots                     &                                &                              \\
\hhline{~=~~}
local data                        &                            &                                & \mrrbrace{5}{caller's frame} \\
\hhline{~-~~}
\mrlbrace{3}{parameter area}      & \ldots                     & \mrrbrace{3}{stack parameters} &                              \\
                                  & \ldots                     &                                &                              \\
                                  & \ldots                     &                                &                              \\
\hhline{~-~~}
                                  & return address             &                                &                              \\
\hhline{~=~~}
local data                        &                            &                                & \mrrbrace{3}{current frame}  \\
\hhline{~-~~}
parameter area                    &                            &                                &                              \\
\hhline{~-~~}
                                  & \vdots                     &                                &                              \\
\hhline{~-~~}
\end{tabular}
\caption{Stack layout on Borland fastcall calling convention}
\end{figure}


\subsubsection{Watcom fastcall}


\paragraph{Registers and register usage}

\begin{table}[h]
\begin{tabular}{3 B}
\hline
Name          & Brief description\\
\hline
{\bf eax}     & scratch, parameter 0, return value@@@\\
{\bf ebx}     & scratch when used for parameter, parameter 2\\
{\bf ecx}     & scratch when used for parameter, parameter 3\\
{\bf edx}     & scratch when used for parameter, parameter 1, return value@@@\\
{\bf esi}     & scratch when used for return pointer @@@??\\
{\bf edi}     & permanent\\
{\bf ebp}     & permanent\\
{\bf esp}     & stack pointer\\
{\bf st0}     & scratch, floating point return value\\
{\bf st1-st7} & scratch\\
\hline
\end{tabular}
\caption{Register usage on Watcom fastcall calling convention}
\end{table}

\paragraph{Parameter passing}

\begin{itemize}
\item stack parameter order: right-to-left
\item called function cleans up the stack
\item first four integers/pointers (\textless=\ 32bit) are passed via eax, edx, ebx and ecx (even if preceded by other arguments@@@?)
\item integer types 64 bits in size @@@ ?
\item all other parameters are pushed onto the stack
\end{itemize}


\paragraph{Return values}

\begin{itemize}
\item return values of pointer or integral type (\textless=\ 32 bits) are returned via the eax register@@@verify, I thnik its esi?
\item integers \textgreater\ 32 bits are returned via the eax and edx registers@@@ verify
\item floating point types are returned via the st0 register@@@ really ?
\end{itemize}


\paragraph{Stack layout}

\begin{figure}[h]
\begin{tabular}{5|3|1 1}
\hhline{~-~~}
                                  & \vdots                     &                                &                              \\
\hhline{~=~~}
local data                        &                            &                                & \mrrbrace{5}{caller's frame} \\
\hhline{~-~~}
\mrlbrace{3}{parameter area}      & \ldots                     & \mrrbrace{3}{stack parameters} &                              \\
                                  & \ldots                     &                                &                              \\
                                  & \ldots                     &                                &                              \\
\hhline{~-~~}
                                  & return address             &                                &                              \\
\hhline{~=~~}
local data                        &                            &                                & \mrrbrace{3}{current frame}  \\
\hhline{~-~~}
parameter area                    &                            &                                &                              \\
\hhline{~-~~}
                                  & \vdots                     &                                &                              \\
\hhline{~-~~}
\end{tabular}
\caption{Stack layout on Watcom fastcall calling convention}
\end{figure}



\subsubsection{win32 stdcall}

\paragraph{Registers and register usage}

\begin{table}[h]
\begin{tabular}{3 B}
\hline
Name          & Brief description\\
\hline
{\bf eax}     & scratch, return value\\
{\bf ebx}     & permanent\\
{\bf ecx}     & scratch\\
{\bf edx}     & scratch, return value\\
{\bf esi}     & permanent\\
{\bf edi}     & permanent\\
{\bf ebp}     & permanent\\
{\bf esp}     & stack pointer\\
{\bf st0}     & scratch, floating point return value\\
{\bf st1-st7} & scratch\\
\hline
\end{tabular}
\caption{Register usage on x86 stdcall calling convention}
\end{table}

\paragraph{Parameter passing}

\begin{itemize}
\item stack parameter order: right-to-left
\item called function cleans up the stack
\item all parameters are pushed onto the stack
\item stack is usually 4 byte aligned (GCC \textgreater=\ 3.x seems to use a 16byte alignement@@@)
\end{itemize}


\paragraph{Return values}

\begin{itemize}
\item return values of pointer or integral type (\textless=\ 32 bits) are returned via the eax register
\item integers \textgreater\ 32 bits are returned via the eax and edx registers
\item floating point types are returned via the st0 register
\end{itemize}


\paragraph{Stack layout}

\begin{figure}[h]
\begin{tabular}{5|3|1 1}
\hhline{~-~~}
                                  & \vdots                     &                                &                              \\
\hhline{~=~~}
local data                        &                            &                                & \mrrbrace{5}{caller's frame} \\
\hhline{~-~~}
\mrlbrace{3}{parameter area}      & \ldots                     & \mrrbrace{3}{stack parameters} &                              \\
                                  & \ldots                     &                                &                              \\
                                  & \ldots                     &                                &                              \\
\hhline{~-~~}
                                  & return address             &                                &                              \\
\hhline{~=~~}
local data                        &                            &                                & \mrrbrace{3}{current frame}  \\
\hhline{~-~~}
parameter area                    &                            &                                &                              \\
\hhline{~-~~}
                                  & \vdots                     &                                &                              \\
\hhline{~-~~}
\end{tabular}
\caption{Stack layout on x86 stdcall calling convention}
\end{figure}



\subsubsection{MS thiscall}

\paragraph{Registers and register usage}

\begin{table}[h]
\begin{tabular}{3 B}
\hline
Name          & Brief description\\
\hline
{\bf eax}     & scratch, return value\\
{\bf ebx}     & permanent\\
{\bf ecx}     & scratch, parameter 0\\
{\bf edx}     & scratch, return value\\
{\bf esi}     & permanent\\
{\bf edi}     & permanent\\
{\bf ebp}     & permanent\\
{\bf esp}     & stack pointer\\
{\bf st0}     & scratch, floating point return value\\
{\bf st1-st7} & scratch\\
\hline
\end{tabular}
\caption{Register usage on MS thiscall calling convention}
\end{table}

\paragraph{Parameter passing}

\begin{itemize}
\item stack parameter order: right-to-left
\item called function cleans up the stack
\item first parameter (this pointer) is passed via ecx
\item all other parameters are pushed onto the stack
\end{itemize}


\paragraph{Return values}

\begin{itemize}
\item return values of pointer or integral type (\textless=\ 32 bits) are returned via the eax register
\item integers \textgreater\ 32 bits are returned via the eax and edx registers@@@verify
\item floating point types are returned via the st0 register@@@ really ?
\end{itemize}


\paragraph{Stack layout}

\begin{figure}[h]
\begin{tabular}{5|3|1 1}
\hhline{~-~~}
                                  & \vdots                     &                                &                              \\
\hhline{~=~~}
local data                        &                            &                                & \mrrbrace{5}{caller's frame} \\
\hhline{~-~~}
\mrlbrace{3}{parameter area}      & \ldots                     & \mrrbrace{3}{stack parameters} &                              \\
                                  & \ldots                     &                                &                              \\
                                  & \ldots                     &                                &                              \\
\hhline{~-~~}
                                  & return address             &                                &                              \\
\hhline{~=~~}
local data                        &                            &                                & \mrrbrace{3}{current frame}  \\
\hhline{~-~~}
parameter area                    &                            &                                &                              \\
\hhline{~-~~}
                                  & \vdots                     &                                &                              \\
\hhline{~-~~}
\end{tabular}
\caption{Stack layout on MS thiscall calling convention}
\end{figure}



\subsubsection{GNU thiscall}

\paragraph{Registers and register usage}

\begin{table}[h]
\begin{tabular}{3 B}
\hline
Name          & Brief description\\
\hline
{\bf eax}     & scratch, return value\\
{\bf ebx}     & permanent\\
{\bf ecx}     & scratch\\
{\bf edx}     & scratch, return value\\
{\bf esi}     & permanent\\
{\bf edi}     & permanent\\
{\bf ebp}     & permanent\\
{\bf esp}     & stack pointer\\
{\bf st0}     & scratch, floating point return value\\
{\bf st1-st7} & scratch\\
\hline
\end{tabular}
\caption{Register usage on x86 GNU thiscall calling convention}
\end{table}

\paragraph{Parameter passing}

\begin{itemize}
\item stack parameter order: right-to-left
\item caller cleans up the stack
\item all parameters are pushed onto the stack
\end{itemize}


\paragraph{Return values}

\begin{itemize}
\item return values of pointer or integral type (\textless=\ 32 bits) are returned via the eax register
\item integers \textgreater\ 32 bits are returned via the eax and edx registers@@@verify
\item floating point types are returned via the st0 register@@@ really ?
\end{itemize}


\paragraph{Stack layout}

\begin{figure}[h]
\begin{tabular}{5|3|1 1}
\hhline{~-~~}
                                  & \vdots                     &                                &                              \\
\hhline{~=~~}
local data                        &                            &                                & \mrrbrace{5}{caller's frame} \\
\hhline{~-~~}
\mrlbrace{3}{parameter area}      & \ldots                     & \mrrbrace{3}{stack parameters} &                              \\
                                  & \ldots                     &                                &                              \\
                                  & \ldots                     &                                &                              \\
\hhline{~-~~}
                                  & return address             &                                &                              \\
\hhline{~=~~}
local data                        &                            &                                & \mrrbrace{3}{current frame}  \\
\hhline{~-~~}
parameter area                    &                            &                                &                              \\
\hhline{~-~~}
                                  & \vdots                     &                                &                              \\
\hhline{~-~~}
\end{tabular}
\caption{Stack layout on x86 GNU thiscall calling convention}
\end{figure}



\subsubsection{pascal}

The best known uses of the pascal calling convention are the 16 bit OS/2 APIs, Microsoft Windows 3.x and Borland Delphi 1.x.

\paragraph{Registers and register usage}

\begin{table}[h]
\begin{tabular}{3 B}
\hline
Name          & Brief description\\
\hline
{\bf eax}     & scratch, return value\\
{\bf ebx}     & permanent\\
{\bf ecx}     & scratch\\
{\bf edx}     & scratch, return value\\
{\bf esi}     & permanent\\
{\bf edi}     & permanent\\
{\bf ebp}     & permanent\\
{\bf esp}     & stack pointer\\
{\bf st0}     & scratch, floating point return value\\
{\bf st1-st7} & scratch\\
\hline
\end{tabular}
\caption{Register usage on x86 pascal calling convention}
\end{table}

\paragraph{Parameter passing}

\begin{itemize}
\item stack parameter order: left-to-right
\item called function cleans up the stack
\item all parameters are pushed onto the stack
\end{itemize}


\paragraph{Return values}

\begin{itemize}
\item return values of pointer or integral type (\textless=\ 32 bits) are returned via the eax register
\item integers \textgreater\ 32 bits are returned via the eax and edx registers
\item floating point types are returned via the st0 register
\end{itemize}


\paragraph{Stack layout}

\begin{figure}[h]
\begin{tabular}{5|3|1 1}
\hhline{~-~~}
                                  & \vdots                     &                                &                              \\
\hhline{~=~~}
local data                        &                            &                                & \mrrbrace{5}{caller's frame} \\
\hhline{~-~~}
\mrlbrace{3}{parameter area}      & \ldots                     & \mrrbrace{3}{stack parameters} &                              \\
                                  & \ldots                     &                                &                              \\
                                  & \ldots                     &                                &                              \\
\hhline{~-~~}
                                  & return address             &                                &                              \\
\hhline{~=~~}
local data                        &                            &                                & \mrrbrace{3}{current frame}  \\
\hhline{~-~~}
parameter area                    &                            &                                &                              \\
\hhline{~-~~}
                                  & \vdots                     &                                &                              \\
\hhline{~-~~}
\end{tabular}
\caption{Stack layout on x86 pascal calling convention}
\end{figure}

