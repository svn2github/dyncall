%//////////////////////////////////////////////////////////////////////////////
%
% Copyright (c) 2007,2008 Daniel Adler <dadler@uni-goettingen.de>, 
%                         Tassilo Philipp <tphilipp@potion-studios.com>
%
% Permission to use, copy, modify, and distribute this software for any
% purpose with or without fee is hereby granted, provided that the above
% copyright notice and this permission notice appear in all copies.
%
% THE SOFTWARE IS PROVIDED "AS IS" AND THE AUTHOR DISCLAIMS ALL WARRANTIES
% WITH REGARD TO THIS SOFTWARE INCLUDING ALL IMPLIED WARRANTIES OF
% MERCHANTABILITY AND FITNESS. IN NO EVENT SHALL THE AUTHOR BE LIABLE FOR
% ANY SPECIAL, DIRECT, INDIRECT, OR CONSEQUENTIAL DAMAGES OR ANY DAMAGES
% WHATSOEVER RESULTING FROM LOSS OF USE, DATA OR PROFITS, WHETHER IN AN
% ACTION OF CONTRACT, NEGLIGENCE OR OTHER TORTIOUS ACTION, ARISING OUT OF
% OR IN CONNECTION WITH THE USE OR PERFORMANCE OF THIS SOFTWARE.
%
%//////////////////////////////////////////////////////////////////////////////

% ==================================================
% PowerPC 32
% ==================================================
\subsection{PowerPC (32bit) Calling Convention}

\paragraph{Overview}

\begin{itemize}
\item word size is 32 bits
\item big endian
\item processor only processes double precision floating point (IEEE-754) values directly (single precision is converted on the fly)
\end{itemize}
For in-depth details about the PowerPC (32bit) calling convention, take a look
at Apple's official documentation \cite{ppcMacOSX} and the Linux Standard Base
(LSB) \cite{ppc32LSB}


\paragraph{\product{dyncall} support}

\product{Dyncall} supports PowerPC (32bit) on Darwin and Linux.


\subsubsection{Darwin (Mac OS X)}

\paragraph{Registers and register usage}

\begin{table}[h]
\begin{tabular}{3 B}
\hline
Name                & Brief description\\
\hline
{\bf gpr0}          & scratch\\
{\bf gpr1}          & stack pointer\\
{\bf gpr2}          & scratch\\
{\bf gpr3}          & return value, parameter 0 if integer or pointer\\
{\bf gpr4-gpr10}    & return value, parameter 1-7 for integer or pointer parameters\\
{\bf gpr11}         & permanent\\
{\bf gpr12}         & branch target for dynamic code generation\\
{\bf gpr13-31}      & permanent\\
{\bf fpr0}          & scratch\\
{\bf fpr1-fpr13}    & parameter 0-12 for floating point (always double precision)\\
{\bf fpr14-fpr31}   & permanent\\
{\bf v0-v1}         & scratch\\
{\bf v2-v13}        & vector parameters\\
{\bf v14-v19}       & scratch\\
{\bf v20-v31}       & permanent\\
{\bf lr}            & scratch, link-register\\
{\bf ctr}           & scratch, count-register\\
{\bf cr0-cr1}       & scratch\\
{\bf cr2-cr4}       & permanent\\
{\bf cr5-cr7}       & scratch\\
\hline
\end{tabular}
\caption{Register usage on ppc32 Darwin}
\end{table}

\paragraph{Parameter passing}

\begin{itemize}
\item stack parameter order: right-to-left@@@?
\item caller cleans up the stack@@@?
\item the first 8 integer parameters are passed in registers gpr3-gpr10
\item the first 12 floating point parameters are passed in registers fpr1-fpr13
\item if a float parameter is passed via a register, gpr registers are skipped for subsequent integer parameters (based on the size of
the float - 1 register for single precision and 2 for double precision floating point values)
\item the caller pushes subsequent parameters onto the stack
\item for every parameter passed via a register, space is reserved in the stack parameter area (in order to spill the parameters if
needed - e.g. varargs)
\item ellipse calls take floating point values in int and float registers (single precision floats are promoted to double precision
as defined for ellipse calls)
\item all nonvector parameters are aligned on 4-byte boundaries
\item vector parameters are aligned on 16-byte boundaries
\item integer parameters \textless\ 32 bit occupy high-order bytes of their 4-byte area
\item composite parameters with size of 1 or 2 bytes occupy low-order bytes of their 4-byte area. INCONSISTENT with other 32-bit PPC
binary interfaces. In AIX and OS 9, padding bytes always follow the data structure
\item composite parameters 3 bytes or larger in size occupy high-order bytes
\end{itemize}


\paragraph{Return values}

\begin{itemize}
\item return values of integer \textless=\ 32bit or pointer type use gpr3
\item 64 bit integers use gpr3 and gpr4 (hiword in gpr3, loword in gpr4)
\item floating point values are returned via fpr1
\item structures \textless=\ 64 bits use gpr3 and gpr4
\item for types \textgreater\ 64 bits, a secret first parameter with an address to the return value is passed
\end{itemize}

\pagebreak

\paragraph{Stack layout}

Stack frame is always 16-byte aligned.\\
\\
\begin{figure}[h]
\begin{tabular}{5|3|1 1}
\hhline{~-~~}
                                  & \vdots              &                                      &                               \\
\hhline{~=~~}
local data                        &                     &                                      & \mrrbrace{13}{caller's frame} \\
\hhline{~-~~}
\mrlbrace{6}{parameter area}      & \ldots              & \mrrbrace{3}{stack parameters}       &                               \\
                                  & \ldots              &                                      &                               \\
                                  & \ldots              &                                      &                               \\
                                  & \ldots              & \mrrbrace{3}{spill area (as needed)} &                               \\
                                  & \ldots              &                                      &                               \\
                                  & gpr3 or fpr1        &                                      &                               \\
\hhline{~-~~}
\mrlbrace{6}{linkage area}        & reserved            &                                      &                               \\
                                  & reserved            &                                      &                               \\
                                  & reserved            &                                      &                               \\
                                  & return address      &                                      &                               \\
                                  & reserved for callee &                                      &                               \\
                                  & saved by callee     &                                      &                               \\
\hhline{~=~~}
local data                        &                     &                                      & \mrrbrace{3}{current frame}   \\
\hhline{~-~~}
parameter area                    &                     &                                      &                               \\
\hhline{~-~~}
linkage area                      & \vdots              &                                      &                               \\
\hhline{~-~~}
\end{tabular}
\caption{Stack layout on ppc32 Darwin}
\end{figure}

